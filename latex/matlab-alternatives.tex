\newpage
\section*{\matlab{} alternatives}
\addcontentsline{toc}{section}{\matlab{} alternatives}

When writing \matlab{} code, you need to realize that unlike C, Fortran, or
Python code, you will always need the \emph{commercial} \matlab{} environment
to have it run. Right now, that might not be much of a problem to you as you
are at a university or have some other free access to the software, but
sometime in the future, this might change.

The current cost for the basic \matlab{} kit, which does not include
\emph{any} toolbox nor Simulink, is €500 for academic institutions;
around €60 for students; \emph{thousands} of Euros for commercial
operations. Considering this, there is a not too small chance that you will
not be able to use \matlab{} after you quit from university, and that would
render all of your own code virtually useless to you.

Because of that, free and open source \matlab{} alternatives have emerged,
three of which are shortly introduced here. Octave and Scilab try to stick to
\matlab{} syntax as closely as possible, resulting in all of the code in this
document being legal for the two packages as well.  When it comes to the
specialized toolboxes, however, neither of the alternatives may be able to
provide the same capabilities that \matlab{} offers. However, these are mostly
functions related to Simulink and the like which are hardly used by beginners
anyway.  Also note none of the alternatives ships with its own text editor (as
\matlab{} does), so you are free yo use the editor of your choice (see, for
example, \href{http://www.vim.org/}{vim},
\href{http://www.gnu.org/software/emacs/}{emacs}, Kate,
\href{http://projects.gnome.org/gedit/}{gedit} for Linux;
\href{http://notepad-plus.sourceforge.net/uk/site.htm}{Notepad++},
\href{http://www.crimsoneditor.com/}{Crimson Editor} for Windows).

\subsection{Python}

\begin{floatingfigure}[r]{6cm}
\centering
\includegraphics[width=6cm]{figures/python-logo-generic}
\end{floatingfigure}

Python is the most modern programming language as of 2013: Amongst the many
award the language as received stands the TIOBE Programming Language Award of
2010. It is yearly given to the programming language that has gained the
largest market market share during that year.

Python is used in all kinds of different contexts, and its versatility and
ease of use has made it attractive to many. There are tons packages for all
sorts of tasks, and the huge community and its open development help the
enormous success of Python.

In the world of scientific computing, too, Python has already risen to be a
major player. This is mostly due to the packages SciPy and Numpy which provide
all data structures and algorithms that are used in numerical code. Plotting
is most easily handled by matplotlib, a huge library which in many ways excels
\matlab{}'s graphical engine.

Being a language rather than an application, Python is supported in virtually
every operating system.

The author of this document highly recommends to take a look at Python for
your own (scientific) programming projects.


\subsection{Julia}

Quoting from \url{julialang.org}:\\
\begin{wrapfigure}{r}{4cm}
  \centering
  \includegraphics[width=4cm]{figures/julia-logo-color.pdf}
\end{wrapfigure}
%\begin{floatingfigure}[r]{4cm}
%  \centering
%  \includegraphics[width=4cm]{figures/julia.png}
%\end{floatingfigure}
\begin{quote}
Julia is a high-level, high-performance dynamic programming language for
technical computing, with syntax that is familiar to users of other technical
computing environments. It provides a sophisticated compiler, distributed
parallel execution, numerical accuracy, and an extensive mathematical function
library. The library, largely written in Julia itself, also integrates mature,
best-of-breed C and Fortran libraries for linear algebra, random number
generation, signal processing, and string processing. In addition, the Julia
developer community is contributing a number of external packages through
Julia’s built-in package manager at a rapid pace. IJulia, a collaboration
between the IPython and Julia communities, provides a powerful browser-based
graphical notebook interface to Julia.
\end{quote}


\subsection{GNU Octave}

\begin{floatingfigure}[r]{5cm}
\centering
\includegraphics[width=5cm]{figures/Octave_Sombrero}
\end{floatingfigure}

GNU Octave is a high-level language, primarily intended for numerical
computations. It provides a convenient command line interface for solving
linear and nonlinear problems numerically, and for performing other numerical
experiments using a language that is mostly compatible with \matlab{}. It may
also be used as a batch-oriented language.

Internally, Octave relies on other independent and well-recognized packages
such as gnuplot (for plotting) or UMFPACK (for calculating with sparse
matrices). In that sense, Octave is extremely well integrated into the free
and open source software (FOSS) landscape.

Octave has extensive tools for solving common numerical linear algebra
problems, finding the roots of nonlinear equations, integrating ordinary
functions, manipulating polynomials, and integrating ordinary differential and
differential-algebraic equations. It is easily extensible and customizable via
user-defined functions written in Octave's own language, or using dynamically
loaded modules written in C++, C, Fortran, or other languages.

GNU Octave is also freely redistributable software. You may redistribute it
and/or modify it under the terms of the GNU General Public License (GPL) as
published by the Free Software Foundation.

The project if originally GNU/Linux, but versions for MacOS, Windows, Sun
Solaris, and OS/2 exist.
